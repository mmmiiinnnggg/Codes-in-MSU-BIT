\documentclass[A4]{article}
\usepackage[T2A]{fontenc}
\usepackage{fontspec}
\setmainfont{CMU Serif}
\usepackage{amsmath}
\usepackage{amssymb}
\usepackage[russian]{babel}
\usepackage{xeCJK}% 调用 xeCJK 宏包
\usepackage{booktabs}
\usepackage{multirow}
\usepackage{graphicx}
\usepackage{listings}
\usepackage{bm}
\usepackage{cancel}
\usepackage{geometry}
\geometry{a4paper,scale=0.9}
\usepackage[usenames,dvipsnames]{xcolor}
\setCJKmainfont{SimHei}

\begin{document}
\author{\textbf{Фамилия, Имя}: Сюй Минчуань\quad \textbf{Номер группы}: $2$ }
\title{Методы оптимизации.\\ Контрольное домашнее задание 2. Вариант 2. }
\maketitle
\noindent\textbf{1.}  Для решения поставленной в пространстве $ L^{2}(0 ; 1) $ задачи минимизации 
\begin{equation*}
J(u)=-\left(\int_{0}^{1} t u(t) d t\right)^{2} \rightarrow \text { inf, } u \in \mathbf{U}=\left\{u=u(t) \in L^{2}(0 ; 1): \int_{0}^{1}\left(u(t)-5 t^{2}\right)^{2} d t \leqslant 1\right\}
\end{equation*}
воспользуйтесь методом штрафных функций, самостоятельно подобрав коэффициенты $ A_{k} $ и число $ p_{1}$. Будет ли сходиться метод штрафных функций? Если да, то будет ли найденный элемент решением задачи?\\
$\boxed{\text{Решение.}}$ Запишем функционал и множество в виде
\begin{equation*}
J(u)=-\langle u,c\rangle^2\rightarrow\inf,\quad u\in\mathbf{U}=\{u=u(t)\in L^2(0;1):g(u)=\|u-f\|^2-1\leqslant 0 \},\quad c=c(t)=t,f=f(t)=5t^2
\end{equation*}
Штраф возьмем
\begin{equation*}
g^+(u)=\max\{g(u),0 \}=\left\{\begin{array}{c}
0,\quad \|u-f\|^2< 1\\
\|u-f\|^2-1,\quad \|u-f\|^2\geqslant 1
\end{array}\right.
\end{equation*}
Теперь штрафная функция и последовательность задач. Здесь берутся $A_k=k,p_1=2$.
\begin{equation*}
P(u)=\left[g^+(u) \right]^2,\quad \Phi_k(u)=\left\{\begin{array}{c}
-\langle u,c\rangle^2+k(\|u-f\|^2-1)^2,\quad \|u-f\|^2\geqslant 1\\
-\langle u,c\rangle^2,\quad\|u-f\|^2< 1
\end{array}\right.
\end{equation*}
Когда $\|u-f\|^2\geqslant 1$, Рассмотрим $\Phi'_k(x)=\Theta$, положив $u=\alpha c+f$.
\begin{equation*}
\Theta=-2\langle u,c\rangle c+4k(u-f)(\|u-f\|^2-1)\quad \Rightarrow\quad 4\alpha^3-(12+\frac{2}{k})\alpha -\frac{15}{2k}=0
\end{equation*}
при $k\rightarrow +\infty$ получаем, $\alpha^3-3\alpha=0$, и $\alpha_1=0,\alpha_{2,3}=\pm \sqrt{3}$. Поэтому $u_1=f,u_{2,3}=\pm\sqrt{3}c+f.$ Проверим, удовлетворяют ли эти точки условию $\|u-f\|^2\geqslant 1$: $\|u_1-f\|^2=0\not\geqslant 1,\|u_{2,3}-f\|^2=3\|c\|^2=1$. Поэтому метод штрафных функций сходится.\\
\textbf{Добавление после консультации:} Покажем, что это действительно решение задачи. Используем неравенство К-Б-:
\begin{equation*}
\langle c,u\rangle^2
\end{equation*} 
\noindent\textbf{2.} В пространстве $ L^{2}(0 ; 2) $ рассматривается задача
\begin{equation*}
J(u)=\int_{0}^{2}(u(t)-t \sqrt{t})^{2} d t+4\left(\int_{0}^{2} \sqrt{t} u(t) d t\right)^{2} \rightarrow \inf _{\mathbf{U}}, \quad \mathbf{U}=\left\{u=u(t) \in L^{2}(0 ; 2):\left(\int_{0}^{2} t \sqrt{t} u(t) d t\right)^{2} \leqslant 1\right\}
\end{equation*}
а) Докажите, что функдионал $ J(u) $ сильно выпуклый, а множество $\mathbf{U}$ выпуклое;\\
б) Докажите, что предлагаемая задача регулярная;\\
в) Решите ее с помощью правила множителей Лагранжа, то есть найдите оптимальное решение $u_*=u_*(t)$ и соответствующие ему множители Лагранжа;\\
г) Проверьте. что найденное оптимальное решение действительно является таковым.\\
$\boxed{\text{Решение.}}$ Запишем функционал и множество виде 
\begin{equation*}
J(u)=\|u-f\|^2+4\langle u,c\rangle^2\rightarrow\inf_{\mathbf{U}},\quad\mathbf{U}=\left\{u=u(t) \in L^{2}(0 ; 2):g(u)=\langle f,u\rangle^2-1\leqslant 0\right\},\quad f=f(t)=t\sqrt{t},c=c(t)=\sqrt{t}
\end{equation*}
a) $J''(u)=2\mathcal{E}+8\langle c,\cdot\rangle c $, $\langle J''(u)h,h\rangle=2\|h\|^2+8\langle c,h\rangle^2\geqslant 2\|h\|^2$. Значит по критерию сильной выпуклости он сильно выпуклый. Далее, пусть $g(u)=\langle f,u\rangle^2-1\leqslant 0$, тогда $\langle g''(u)h,h\rangle=2\langle f,h\rangle^2\geqslant 0$, значит по критерию выпуклости $g(u)$ выпуклый, а значит множество $\mathbf{U}$ выпукло по первому контрольному заданию.\\
б) Для регулярности задачи достаточно проверить условие Слейтера, в качестве Слейтеровой точки возьмем $u_0=\Theta$. Понятно, что $g(u_0)=-1<0$ и $u_0\in L^2(0;2)$, поэтому выполнено условие Слейтера.\\
в) Запишем функцию Лагранжа:
\begin{equation*}
\mathcal{L}(u,\lambda)=\|u-f\|^2+4\langle u,c\rangle^2+\lambda\langle f,u\rangle^2-\lambda.\quad u\in L^2(0;2),\lambda\in\mathbb{R}.
\end{equation*}
По условию задачи и вышедоказанным, $\mathbf{U}_0 = L^2(0;2)$ - выпуклое, $J(u)$ - сильно выпуклый, $g(u)$ - выпуклый, задача регулярная, значит применима теорема Куна-Таккера, и надо найти решение $(u_*,\lambda^*)$ из системы
\begin{equation*}
\left\{\begin{array}{c}
\mathcal{L}(u_*,\lambda^*)\leqslant\mathcal{L}(u,\lambda^*);\\
\lambda^*g(u)=0;\\
\lambda^*\geqslant 0.
\end{array}\right.
\end{equation*}
удовлетворяющее $u_*\in\mathbf{U}$. При всех $\lambda^*\geqslant 0 $ функция Лагранжа выпукла и дифференцируема, поэтому первое неравенство равносильно $\mathcal{L}_u(u_*,\lambda^*)=\Theta$, поэтому
\begin{equation*}
\left\{\begin{array}{c}
2(u_*-f)+8\langle u_*,c\rangle c+2\lambda\langle u_*,f\rangle f=\Theta\\
\lambda^*(\langle f,u\rangle^2-1)=0;\\
\lambda^*\geqslant 0.
\end{array}\right.
\end{equation*}
Решим первое уравнение. Преобразуем и получаем
\begin{equation*}
u_*=-4\langle u_*,c\rangle c+(1-\lambda\langle u_*,f\rangle) f
\end{equation*}
Заметим, что $u_*$ представляет собой линейную комбинацию элемента $c$ и $f$, поэтому положим $u_*=\mu c+\upsilon f$. Подставляя в уравнение и получаем
\begin{equation*}
\begin{aligned}
\mu c+\upsilon f=&-4\langle \mu c+\upsilon f,c\rangle c+(1-\lambda\langle \mu c+\upsilon f,f\rangle) f=-4(\mu\|c\|^2+\upsilon\langle f,c\rangle)c+(1-\lambda(\mu\langle f,c\rangle+\upsilon\|f\|^2))f=\\
=&\textcolor{blue}{ \{\|c\|^2=\int_{0}^{2}tdt=2,\|f\|^2=\int_{0}^{2}t^3dt=4,\langle f,c\rangle=\int_{0}^{2}t^2dt=\frac{8}{3} \} } =-4(2\mu+\frac{8}{3}\upsilon)c+(1-\lambda(\frac{8}{3}\mu+4\upsilon))f\Rightarrow\\
\Rightarrow&(8\mu+\frac{32}{3}\upsilon+\mu)c+(\frac{8}{3}\mu\lambda+4\upsilon\lambda-1+v)f=\Theta\quad\Rightarrow\\
&\left\{\begin{array}{c}
9\mu+\frac{32}{3}\upsilon=0\\
\frac{8}{3}\lambda\mu+(4\lambda+1)\upsilon=1
\end{array}\right.\quad\Rightarrow\quad\left\{\begin{array}{c}
\mu=-\frac{96}{68\lambda+81}\\
\upsilon=\frac{81}{68\lambda+81}
\end{array}\right.
\end{aligned}
\end{equation*}
Поэтому $u=-\frac{96}{68\lambda+81}c+\frac{81}{68\lambda+81}f$. Подставляя в второе уравнение, рассмотрим два случая:\\
\textbf{1.} $\bm{\lambda =0}$ Тогда $2(u-f)+8\langle u,c\rangle c=\Theta$. Опять положим $u=ac+bf$, получаем
\begin{equation*}
\begin{aligned}
&2(ac+bf-f)+8\langle ac+bf,c\rangle c=\Theta\quad\Rightarrow\quad (18a+\frac{8}{3}b)c+(2b-2)f=\Theta\quad\Rightarrow\\
&\left\{\begin{array}{c}
18a+\frac{8}{3}b=0\\
2b-2=0
\end{array}\right.\quad\Rightarrow\quad\left\{\begin{array}{c}
a=-\frac{4}{27}\\
b=1
\end{array}\right.\quad\Rightarrow\quad u=-\frac{4}{27}c+f
\end{aligned}
\end{equation*}
Проверим, принадлежажит ли он множеству $\mathbf{U}$:
\begin{equation*}
\langle f,u\rangle^2-1=(-\frac{4}{27}\langle c,f\rangle+\|f\|^2)^2-1=(-\frac{4}{27}\times\frac{8}{3}+4)^2-1\not\leqslant 0
\end{equation*}
Поэтому найденное решение не лежит в $\mathbf{U}$.\\
\textbf{2.} $\bm{\lambda \ne 0}$ Тогда 
\begin{equation*}
\begin{aligned}
&\langle f,\mu c+\upsilon f\rangle^2-1=0\quad\Rightarrow\quad (\mu\langle f,c\rangle+\upsilon\|f\|^2)^2=1\quad\Rightarrow\quad (\frac{8}{3}\mu+4\upsilon)^2=1\quad\Rightarrow\quad \frac{8}{3}\mu+4\upsilon=\pm 1\\
&\frac{8}{3}\left(-\frac{96}{68\lambda+81} \right)+\frac{4\cdot 81}{68\lambda+81}=\pm 1\quad\Rightarrow\quad \frac{68}{68\lambda+81}=\pm 1\quad\Rightarrow\quad \lambda_1=-\frac{13}{68},\lambda_2=-\frac{149}{68}
\end{aligned}
\end{equation*}
\noindent\textbf{3.} В пространстве $ \ell^{2} $ рассматривается задача
\begin{equation*}
\begin{aligned}
J(x)&=\left(\sum_{n=1}^{\infty} \frac{x_{n}}{2^{n-2}}\right) \cdot\left(\sum_{n=1}^{\infty} \frac{x_{n}}{3^{n-1}}\right) \rightarrow \inf _{\mathbf{X}} \\
\mathbf{X}&=\left\{x=\left(x_{1}, x_{2}, \ldots, x_{n}, \ldots\right) \in \ell^{2}: \sum_{n=1}^{\infty} x_{n}^{2} \leqslant 10\right\}
\end{aligned}
\end{equation*}
а) Покажите, что эта задача не является выпуклой;\\
б) Покажите, что предлагаемая задача регулярная;\\
в) Решите ее с помощью правила множителей Лагранжа, то есть найдите оптимальное решение $ x_{*}$ и соответствуюшие ему множители Лагранжа;\\
г) Выпишите и решите двойственную к ней задачу;\\
п) Проведите обоснование того, что найденное оптимальное решение действительно является таковым.\\
$\boxed{\text{Решение.}}$ Запишем функционал и множество виде 
\begin{equation*}
\begin{aligned}
J(x)&=\langle x,c\rangle\langle x,d\rangle\rightarrow\inf_{\mathbf{X}},\quad\mathbf{X}=\left\{x=(x_1,x_2,\ldots,x_n,\ldots)\in\ell^2:g(x)=\|x\|^2-10\leqslant 0\right\}\\
c&=(\frac{1}{2^{-1}},\frac{1}{2^0},\ldots,\frac{1}{2^{n-2}},\ldots),\quad d=(\frac{1}{3^{0}},\frac{1}{3^1},\ldots,\frac{1}{3^{n-1}},\ldots)
\end{aligned}
\end{equation*}
a) Задача не является выпуклой, потому что функционал $J(u)$ не выпуклый:
\begin{equation*}
J''(x)=\langle c,\cdot\rangle d+\langle d,\cdot\rangle c\quad\Rightarrow\quad \langle J''(x)h,h\rangle=2\langle c,h\rangle \langle d,h \rangle \quad\forall h\in\mathbb{H}
\end{equation*}
Понятно, что $\langle J''(x)h,h\rangle$ не всегда неотрицательны. Для этого достаточно рассмотреть $h=(-1,\frac{5}{2},0,0,\ldots)$. Понятно что,
\begin{equation*}
2\langle c,h\rangle \langle d,h \rangle=2(2\cdot(-1)+1\cdot\frac{5}{2})\cdot(1\cdot(-1)+\frac{1}{3}\cdot\frac{5}{2})=-\frac{1}{6}<0
\end{equation*}
б) Поскольку задача не выпукла, и в ней нет ограничения-равества, то достаточно проверить условие
\begin{equation*}
\exists h\in\ell^2: \langle g'(x_*),h\rangle <0
\end{equation*}
где $x_*$ - точка минимума. Если $x_*\ne\Theta$, то можно положить $h=(0,\ldots,0,-x_i,0,\ldots,0)$, где $x_i$ - координата элемента $x_*$, не равная нулю. Координата $-x_i$ стоит в $i$-ом месте. А если $x_*=\Theta$, то $J(x_*)=J_*=0$. Но $J(x)$ может принимать отрицательное значение, если возьмем $x=(-1,\frac{5}{2},0,0,\ldots)$ как в пункте а.\\
в) Запишем функцию Лагранжа:
\begin{equation*}
\mathcal{L}(x,\lambda)=\langle x,c\rangle\langle x,d\rangle+\lambda(\|x\|^2-10)\quad x\in\ell^2,\lambda\in\mathbb{R}.
\end{equation*}
Условие регулярности проведено в пунке б. Далее напишем необходимое условие локального минимума:
 \begin{equation*}
 \left\{\begin{array}{c}
\langle c,x\rangle d+\langle d,x\rangle c+2\lambda^* x=\Theta\\
 \lambda^*(\|x\|^2-10)=0;\\
 \lambda^*\geqslant 0.
 \end{array}\right.
 \end{equation*}
 Решим первое из них: положим $x=\mu c+\upsilon d$. Подставляя в уравнение получаем
 \begin{equation*}
 \begin{aligned}
\Theta=&\langle c,\mu c+\upsilon d\rangle d+\langle d,\mu c+\upsilon d\rangle c+2\lambda x=(\mu\|c\|^2+\upsilon \langle c,d\rangle)d+(\mu \langle d,c\rangle+\upsilon\|d\|^2)c+2\lambda(\mu c+\upsilon d)=\\
 =&\textcolor{blue}{ \{\|c\|^2=\sum_{n=1}^{\infty}\frac{1}{4^{n-2}}=\frac{16}{3},\|d\|^2=\sum_{n=1}^{\infty}\frac{1}{9^{n-1}}=\frac{9}{8},\langle c,d\rangle=\sum_{n=1}^{\infty}\frac{1}{2^{n-2}}\cdot\frac{1}{3^{n-1}}=2\sum_{n=1}^{\infty}\frac{1}{6^{n-1}}=\frac{12}{5} \} }=\\
 =&(\frac{16}{3}\mu+\frac{12}{5}\upsilon)d+(\frac{12}{5}\mu+\frac{9}{8}\upsilon)c+2\lambda\mu c+2\lambda\upsilon d\quad\Rightarrow\\
 &\left\{\begin{array}{c}
 (\frac{12}{5}+2\lambda)\mu+\frac{9}{8}\upsilon=0\\
 \frac{16}{3}\mu+(\frac{12}{5}+2\lambda)\upsilon=0
 \end{array}\right.\Rightarrow Au=\Theta,u=(\mu,\upsilon)^T
 \end{aligned}
 \end{equation*}
 Понятно, что опредилитель матрицы $A$ должен равен нулю, если требуется существование нетривиального решения (тривиальное решение приводит к нулевому элементу $x=\Theta$, что не является точкой минимума.) Поэтому
 \begin{equation*}
 (\frac{12}{5}+2\lambda)^2-\frac{9}{8}\cdot\frac{16}{3}=0\quad\Rightarrow\quad \frac{12}{5}+2\lambda=\pm\sqrt{6},\quad\lambda_1=\frac{\sqrt{6}}{2}-\frac{6}{5},\lambda_2= -\frac{\sqrt{6}}{2}-\frac{6}{5}
 \end{equation*}
 $\lambda_2<0$, поэтому исключаем из рассмотрения. Подставляя $\lambda_1$ во второе уравнение, получаем $\|x\|^2=10$, то есть все точки на границы множества $\mathbf{X}$ являются подозрительными точками минимума.\\
 Далее, пусть $k=\frac{12}{5}+2\lambda=\sqrt{6}$, и из системы уравнений про $\mu,\upsilon$ следует, что $\upsilon=-\frac{8k}{9}\mu$. Подставлять его в выражение для $\|x\|^2=10$, напоминая $x=\mu c+\upsilon d$, получаем
\begin{equation*}
\begin{aligned}
&\|x\|^2=\|\mu c+\upsilon d\|^2=\|\mu c-\frac{8k}{9}\mu d\|^2=\mu^2\|c-\frac{8}{9}kd\|^2=\mu^2(\|c\|^2+\frac{64}{81}k^2\|d\|^2-\frac{16}{9}k\langle c,d\rangle)=\mu^2(\frac{8}{9}k^2-\frac{64}{15}k+\frac{16}{3})=10\\
&\mu_{1,2}=\pm\sqrt{\frac{150}{160-64\sqrt{6}}}\quad\Rightarrow\quad x_{1,2}=\mu c-\frac{8\sqrt{6}}{9}\mu d
\end{aligned}
\end{equation*}
Вычисляем значения функционала в этих точках: сначала преобразуем $J(x)$
\begin{equation*}
J(x)=\langle x,c\rangle\langle x,d\rangle=\langle c,\mu c+\upsilon d\rangle d+\langle d,\mu c+\upsilon d\rangle c=\frac{64}{5}\mu^2+\frac{27}{10}\upsilon^2+\frac{294}{25}\mu\upsilon=\{\upsilon=-\frac{8k\mu}{9} \}=\left(\frac{32}{15}k^2-\frac{784}{75}k+\frac{64}{5}\right)\mu^2
\end{equation*}
Тогда
\begin{equation*}
J(x)=J(\mu)=\left(\frac{32}{15}k^2-\frac{784}{75}k+\frac{64}{5}\right)\cdot\frac{10}{\frac{8}{9}k^2-\frac{64}{15}k+\frac{16}{3}}=\text{CONST}
\end{equation*}
Поэтому, подозрительными точками минимума являются $x_{1}$ и $x_2$.\\
г) Выпишем двойственную задачу:
\begin{equation*}
\Psi(\lambda)=\inf\limits_{x\in\mathbf{X}}\mathcal{L}(x,\lambda)\rightarrow\sup\limits_{\lambda\in[0;+\infty)}
\end{equation*}
Попробуем решить в ней задачу минимизации. Это практически совпадает с пунктом в):
\begin{equation*}
\Theta=\mathcal{L}'(x,\lambda)=\langle c,x\rangle d+\langle d,x\rangle c+2\lambda x
\end{equation*}
Значит, по пункту в):\\
1. Если $\lambda=\frac{\sqrt{6}}{2}-\frac{6}{5}$, то подозрительный минимум не равен $\Theta$.\\
2. Если $\lambda\ne \frac{\sqrt{6}}{2}-\frac{6}{5}$, то подозрительный минимум равен $\Theta$, тогда $\mathcal{L}(\Theta,\lambda)=-10\lambda$.\\
Итак
\begin{equation*}
\Psi(\lambda)=\inf\limits_{x\in\mathbf{X}}\mathcal{L}(x,\lambda)= \left\{\begin{array}{c}
-10\lambda,\quad \lambda\ne \frac{\sqrt{6}}{2}-\frac{6}{5}\\
k(\mu),\quad \lambda= \frac{\sqrt{6}}{2}-\frac{6}{5}
\end{array}\right.\rightarrow\sup\limits_{\lambda\in[0;+\infty)}
\end{equation*}
где $k(\mu)=\langle x(\mu),c\rangle\langle x(\mu),d\rangle+\lambda(\|x(\mu)\|^2-10)$.\\
\end{document}

