\documentclass[A4]{article}
\usepackage[T2A]{fontenc}
\usepackage{fontspec}
\setmainfont{CMU Serif}
\usepackage{amsmath}
\usepackage{amssymb}
\usepackage[russian]{babel}
\usepackage{xeCJK}% 调用 xeCJK 宏包
\usepackage{booktabs}
\usepackage{multirow}
\usepackage{graphicx}
\usepackage{listings}
\usepackage{bm}
\usepackage{cancel}
\usepackage{geometry}
\geometry{a4paper,scale=0.9}
\usepackage[usenames,dvipsnames]{xcolor}
\setCJKmainfont{SimHei}

\begin{document}
\author{\textbf{Фамилия, Имя}: Сюй Минчуань\quad \textbf{Номер группы}: $2$ }
\title{Методы оптимизации.\\ Контрольное домашнее задание 1. Вариант 2. }
\maketitle
\noindent\textbf{1.1.} Исследуйте на непрерывность, полунепрерывность снизу, выпуклость, слабую полунепрерывность снизу функционал $J: \ell^{2} \rightarrow \mathbb{R}$ 
\begin{equation*}
J(x)=\left(\sum_{n=1}^{\infty} \frac{x_{n}}{n^{4}}\right)^{2}-\sum_{n=1}^{\infty} \frac{x_{n}}{n+\ln n}
\end{equation*}
Будет ли он слабо непрерывным?\\
\\
$\boxed{\text{Решение.}}$ Пусть $u=(x_1,x_2,\ldots,x_n,\ldots)\in\ell^2,x=(\frac{1}{1^4},\frac{1}{2^4},\ldots,\frac{1}{n^4},\ldots),y=(\frac{1}{1+\ln n},\frac{1}{2+\ln n},\ldots,\frac{1}{n+\ln n},\ldots)$. $x,y\in\ell^2$ так как $\sum\limits_{n=1}^{\infty}\frac{1}{n^4}<+\infty,\sum\limits_{n=1}^{\infty}\frac{1}{n+\ln n}<\sum\limits_{n=1}^{\infty}\frac{1}{n^2}<+\infty$. Запишем функционал в виде: $J(u)=\langle u,x\rangle^2-\langle u,y\rangle.$ Он будет слабо непрерывный, так как суперпозиция вещественной функции от слабо непрерывного функционала $f(J(u))$ и разность  слабо непрерывных функционалов $J(u)-K(u)$ - слабо непрерывный, а функционал вида $\langle c,u\rangle,c\in\mathbb{H}$ - слабо непрерывный. \\
Далее, из слабо непрерывности следует непрерывность, слабо полунепрерывность снизу и полунепрерывность снизу.\\
Для выпуклости доказывается по определению: $J\left(\alpha u_{1}+(1-\alpha) u_{2}\right) \leqslant \alpha J\left(u_{1}\right)+(1-\alpha) J\left(u_{2}\right) \quad \forall u_{1}, u_{2} \in\ell^2, \alpha \in[0;1]$.
Поэтому
\begin{equation*}
\langle \alpha u_1+(1-\alpha)u_2,x\rangle^2-\langle  \alpha u_1+(1-\alpha)u_2,y\rangle\leqslant  \alpha\langle u_1,x\rangle^2-\alpha\langle u_1,y\rangle+(1-\alpha)\langle u_2,x\rangle^2-(1-\alpha)\langle u_2,y\rangle
\end{equation*}
\begin{equation*}
\begin{aligned}
\langle \alpha u_1+(1-\alpha)u_2,x\rangle^2-\cancel{\alpha\langle u_1,y\rangle}- \cancel{(1-\alpha)\langle u_2,y\rangle}\leqslant&  \alpha\langle u_1,x\rangle^2-\cancel{\alpha\langle u_1,y\rangle}+(1-\alpha)\langle u_2,x\rangle^2-\cancel{(1-\alpha)\langle u_2,y\rangle}\\
\big[\alpha\langle u_1,x\rangle+(1-\alpha)\langle u_2,x\rangle \big]^2\leqslant& \alpha\langle u_1,x\rangle^2+(1-\alpha)\langle u_2,x\rangle^2\\
\big(\alpha(\alpha-1)\big)\langle u_1,x\rangle^2+\big(\alpha(\alpha-1)\big)\langle u_2,x\rangle^2+&2\alpha(1-\alpha)\langle u_1,x\rangle\langle u_2,x\rangle\leqslant 0\\
\alpha(1-\alpha)\big[\langle u_1,x\rangle-\langle u_2,x\rangle\big]^2\geqslant& 0\\
\end{aligned}
\end{equation*}
В силу $\alpha(1-\alpha)\geqslant0$ и неотрицательности квадрата получаем правильное неравенство, что и верно исходное.\\
$\boxed{\text{Ответ.}}$ Функционал непрерывен, полунепрерывен снизу, выпукл, слабо полунепрерывен снизу и слабо непрерывен.\\
\\
\textbf{1.2.} В евклидовом пространстве $\mathbb{H}$ дано непустое множество $\mathbf{U}=\{u\in\mathbb{H}:\quad g_1(u)\leqslant 0,\ldots,g_m(u)\leqslant 0 \}$. \\
Докажите, что:\\
а) если все функционалы $ g_{i}(u)$ полунепрерывны (слабо полунепрерывным) снизу на $ \mathbb{H}$, то множество $ \mathbf{U}$ замкнуто (слабо замкнуто);\\
б) если все функционалы $ g_{i}(u) $ выпуклы на $ \mathbb{H}$, то множество $ \mathbf{U} $ выпукло.\\
\\
$\boxed{\text{Решение.}}$ 
а) Пусть последовательность $\{u_n \},u_i\in\mathbf{U}, i=1,2,\ldots,m$ сильно (слабо) сходиться к некоторой точке $u_0\in\mathbb{H}$. По определению (слабо) полунепрерывности снизу и свойству нижнего предела последовательности
\begin{equation*}
0\geqslant\varliminf_{n\rightarrow\infty} g_i(u_n)\geqslant g_i(u_0),\quad i=1,2,\ldots,m
\end{equation*}
Значит $u_0$ тоже принадлежит множеству $\mathbf{U}$. Это и есть определение (слабо) замкнутого множества.\\
б) Рассмотрим любые точки $u\in\mathbf{U},v\in\mathbf{U}$. Они являются точками этого множества, поэтому выполнены
\begin{equation*}
g_1(u)\leqslant 0,g_2(u)\leqslant 0,\ldots,g_m(u)\leqslant 0;\quad g_1(v)\leqslant 0,g_2(v)\leqslant 0,\ldots,g_m(v)\leqslant 0. 
\end{equation*}
Рассмотрим точку $\alpha u+(1-\alpha)v,\alpha\in[0;1]$. В силу выпуклости функционалов $g_i(u)$ имеем
\begin{equation*}
g_i(\alpha u+(1-\alpha)v)\leqslant\alpha \underbrace{g_i(u)}_{\leqslant 0}+(1-\alpha)\underbrace{g_i(v)}_{\leqslant 0}\leqslant 0,\quad i=1,2,\ldots,m
\end{equation*}
Поэтому точку $\alpha u+(1-\alpha)v\in\mathbf{U},\alpha\in[0;1]$. Значит, множество $\mathbf{U}$ - выпуклый.\\
$\boxed{\text{Ответ.}}$ Доказательства см. выше\\
\\
\textbf{1.3.} В пространстве $ L^{2}(0 ; 1) $ рассматривается множество
\begin{equation*}
\mathbf{U}=\left\{u=u(t) \in L^{2}(0 ; 1):\left(\int_{0}^{1} u^{2}(t) d t\right)^{2} \leqslant 9+\left(\int_{0}^{1} t^{2} u(t) d t\right)^{2}\right\}
\end{equation*}
Исследуйте его на выпуклость, замкнутость, ограниченность, слабую компактность, компактность.\\
$\boxed{\text{Решение.}}$ Запишем определение множества в виде
\begin{equation*}
\mathbf{U}=\left\{u=u(t) \in L^{2}(0 ; 1): J(u)=\|u|^4-\langle u,c\rangle^2\leqslant 9\right\},\quad c=c(t)=t^2
\end{equation*}
1) \underline{Выпуклость.} По определению множества имееем
\begin{equation*}
\begin{aligned}
&J(\alpha u_1+(1-\alpha)u_2)=\|\alpha u_1+(1-\alpha)u_2\|^4-\langle\alpha u_1+(1-\alpha)u_2,c\rangle^2\leqslant 9\\
&\alpha J(u_1)=\alpha \|u_1\|^4-\alpha u_1,c\rangle^2\leqslant 9\alpha\\
&(1-\alpha)J(u_2)=(1-\alpha) \|u_2\|^4-(1-\alpha) u_2,c\rangle^2\leqslant 9(1-\alpha)
\end{aligned}
\end{equation*} 
Теперь запишем
\begin{equation*}
\begin{aligned}
&J(\alpha u_1+(1-\alpha)u_2)-\alpha J(u_1)-(1-\alpha)J(u_2)=\\
&=\|\alpha u_1+(1-\alpha)u_2\|^4-\langle\alpha u_1+(1-\alpha)u_2,c\rangle^2-\alpha \|u_1\|^4+\alpha \langle u_1,c\rangle^2-(1-\alpha) \|u_2\|^4+(1-\alpha) u_2,c\rangle^2\leqslant\\
&\leqslant 9-\alpha \|u_1\|^4+\alpha \langle u_1,c\rangle^2-(1-\alpha) \|u_2\|^4+(1-\alpha) u_2,c\rangle^2\leqslant 9-9\alpha -9+9\alpha =0
\end{aligned}
\end{equation*}
В итоге получаем $J(\alpha u_1+(1-\alpha)u_2)\leqslant\alpha J(u_1)+(1-\alpha)J(u_2),\forall u_1,u_2\in\mathbf{U}.$ Это определение выпуклого множества.\\
2) \underline{Замкнутость.} Функционал $J(u)$ непрерывен как разность непрерывных функционалов. Из непрерывности следует полунепрерывность снизу, а по первому утверждению из задачи $1.2$ видно, что множество замкнуто.\\
3) \underline{Ограниченность.} Пусть множество $\mathbf{U}_0=\{ u=u(t)\in L^2(0;1): \|u\|^4\leqslant 9+\|u\|^2\|c\|^2\},\quad c=c(t)=t^2$. Видно, что $\mathbf{U}\subseteq\mathbf{U}_0$, поскольку все точки, принадлежащие множеству $\mathbf{U}$, также принадлежат множеству $\mathbf{U}_0$ (Это потому что выполнено неравенство Коши-Буняковского: $\langle u,c\rangle^2\leqslant\|u\|^2\|c\|^2$). Далее, можно записать неравенства для $\mathbf{U}_0$ в виде:
\begin{equation*}
\|u\|^2(\|u\|^2-\|c\|^2)\leqslant 9
\end{equation*}
Если смотреть $\|u\|$ как переменную, то по графику функции можно увидеть, что если эта функция меньше какого-то постоянного, то множество точек, удовлетворяющих этому ограничению будет ограниченным, то есть $\|u\|\leqslant\text{const}$ - это и есть определение ограниченного множества.\\
4) \underline{Слабая компактность.} Из выпуклости, замкнутости и ограниченности по достаточному условию слабо компактности следует требуемое свойство.\\
5) \underline{Компактность.} Это множество не компактно. Рассмотрим ортонормированную систему $\{e_n \}$. Из ней нельзя выделить сходящуюся подпоследовательность.\\
$\boxed{\text{Ответ.}}$ Множество выпукло, замкнуто, ограничено, слабо компактно и компактно.\\
\textbf{2.1.} Вычислите градиент и гессиан функционала $J(x)=\cos \left(\sum\limits_{n=1}^{\infty}\left(2 x_{n}-x_{n+1}-\frac{1}{n}\right)^{2}\right): \ell^{2} \rightarrow \mathbb{R}^{1}.$\\
$\boxed{\text{Решение.}}$ Запишем функционал в виде $J(u)=\cos\|\mathcal{A}u-f\|^2$, где
\begin{equation*}
 u=(x_1,x_2,\ldots,x_n,\ldots)\in\ell^2,\mathcal{A}u=(2x_1-x_2,2x_2-x_3,\ldots,2x_n-x_{n+1},\ldots),f=(\frac{1}{1},\frac{1}{2},\ldots,\frac{1}{n},\ldots).
\end{equation*}
Пусть $\mathcal{A}u=\mathcal{B}u-\mathcal{C}u$, где $\mathcal{B}$ - оператор умножения, $\mathcal{C}$ - оператор правого сдвига, то есть 
\begin{equation*}
\mathcal{B}u=(2x_1,2x_2,\ldots,2x_n,\ldots),\quad \mathcal{C}u=(x_2,x_3,\ldots,x_n,x_{n+1},\ldots).
\end{equation*}
Оператор $\mathcal{B}$ самосопряженный, так как $\forall u=(x_1,\ldots,x_n,\ldots),v=(y_1,\ldots,y_n,\ldots)$
\begin{equation*}
(\mathcal{B}u,v)=\sum_{n=1}^{\infty}2x_n\cdot y_n =\sum_{n=1}^{\infty} x_n\cdot 2y_n=(u,\mathcal{B}^*v)\Rightarrow\mathcal{B}^*v=(2y_1,2y_2,\ldots,2y_n,\ldots)\Rightarrow \mathcal{B}=\mathcal{B}^*
\end{equation*}
Оператор $\mathcal{C}^*$ - оператор левого сдвига, так как $\forall u=(x_1,\ldots,x_n,\ldots),v=(y_1,\ldots,y_n,\ldots)$
\begin{equation*}
\begin{aligned}
(\mathcal{C}u,v)=&x_2y_1+x_3y_2+\ldots+x_{n+1}y_n+\ldots=x_1\cdot\underbrace{y_0}_{=0}+x_2y_1+x_3y_2+\ldots+x_{n+1}y_n+\ldots=(u,\mathcal{C}^*v)\\
\mathcal{C}^*v=&(0,y_1,y_2,\ldots,y_n,\ldots)
\end{aligned}
\end{equation*}
Поэтому в силу линейности сопряженного оператора $\mathcal{A}^*u=\mathcal{B}^*u-\mathcal{C}^*u=(2x_1,2x_2-x_1,\ldots,2x_{n+1}-x_n,\ldots)$.
\begin{equation*}
\begin{aligned}
J'(u)=&-\sin \|\mathcal{A}u-f\|^2\cdot 2\mathcal{A}^*(\mathcal{A}u-f),\\
J''(u)=&-\cos\|\mathcal{A}u-f\|^2\cdot 2\mathcal{A}^*(\mathcal{A}u-f)\cdot 2\mathcal{A}^*(\mathcal{A}u-f)-\sin \|\mathcal{A}u-f\|^2\cdot 2\mathcal{A}^*\mathcal{A}
\end{aligned}
\end{equation*}
$\boxed{\text{Ответ.}}$\\
\begin{equation*}
\begin{aligned}
J'(u)=&-2\sin \|\mathcal{A}u-f\|^2\mathcal{A}^*(\mathcal{A}u-f),\\
J''(u)=&-4\cos\|\mathcal{A}u-f\|^2 \big\langle\mathcal{A}^*(\mathcal{A}u-f),\cdot\big\rangle \mathcal{A}^*(\mathcal{A}u-f)-2\sin \|\mathcal{A}u-f\|^2\cdot \mathcal{A}^*\mathcal{A}
\end{aligned}
\end{equation*}
где 
\begin{equation*}
u=(x_1,\ldots,x_n,\ldots)\in\ell^2,\mathcal{A}u=(2x_1-x_2,\ldots,2x_n-x_{n+1},\ldots),f=(\frac{1}{1},\ldots,\frac{1}{n},\ldots),\mathcal{A}^*u=(2x_1,2x_2-x_1,\ldots,2x_{n+1}-x_n,\ldots)
\end{equation*}
\\
\textbf{2.2.} Вычислите градиент и гессиан функционала $ J(u)=\left(\int\limits_{0}^{\pi} u(t) \sin t d t\right)^{4} \cdot\left(\int\limits_{0}^{\pi} u(t) u(\pi-t) d t\right): L^{2}(0 ; \pi) \rightarrow \mathbb{R}^{1}.$\\
$\boxed{\text{Решение.}}$ Запишем функционал в виде $J(u)=\langle u,c\rangle^4\cdot\langle\mathcal{A}u,u\rangle,c=\sin t,\mathcal{A}u(t)=u(\pi-t)$. Покажем, что $\mathcal{A}=\mathcal{A}^*$:
\begin{equation*}
\begin{aligned}
\forall u=&u(t),v=v(t)\in L^2(0;\pi)\quad\langle \mathcal{A}u,v\rangle=\int_{0}^{\pi}u(\pi-t)v(t)dt=\textcolor{blue}{\{\pi-t=x,dt=-dx,0\rightleftarrows\pi\}}=\int_{0}^{\pi}u(x)v(\pi-x)dx=\textcolor{blue}{\{t\rightleftarrows x\}}=\\
=&\int_{0}^{\pi}u(t)v(\pi-t)dt=\langle u,\mathcal{A}^*v\rangle,\quad \mathcal{A}^*v(t)=v(\pi-t)
\end{aligned}
\end{equation*}
Теперь можно спокойно вычислять производные:
\begin{equation*}
\begin{aligned}
J'(u)=&4\langle u,c\rangle^3c\langle \mathcal{A}u,u\rangle+\langle u,c\rangle^4(\mathcal{A}+\mathcal{A}^*)u=4\langle u,c\rangle^3\langle \mathcal{A}u,u\rangle c+2\langle u,c\rangle^4\mathcal{A}u.\\
J''(u)=&12\langle u,c\rangle^2\langle \mathcal{A}u,u\rangle\langle c,\cdot\rangle c+8\langle u,c\rangle^3\langle \mathcal{A}u,\cdot\rangle c+8\langle u,c\rangle^3\langle c,\cdot\rangle \mathcal{A}u+2\langle u,c\rangle^4\mathcal{A}
\end{aligned}
\end{equation*}
 $\boxed{\text{Ответ.}}$\\
 \begin{equation*}
 \begin{aligned}
 J'(u)=&4\langle u,c\rangle^3\langle \mathcal{A}u,u\rangle c+2\langle u,c\rangle^4\mathcal{A}u.\\
 J''(u)=&12\langle u,c\rangle^2\langle \mathcal{A}u,u\rangle\langle c,\cdot\rangle c+8\langle u,c\rangle^3\langle \mathcal{A}u,\cdot\rangle c+8\langle u,c\rangle^3\langle c,\cdot\rangle \mathcal{A}u+2\langle u,c\rangle^4\mathcal{A}\\
 \text{где}& \quad\mathcal{A}u(t)=u(\pi-t),c=\sin t.
 \end{aligned}
 \end{equation*}
 \\
\textbf{3.1.} Пусть функционал $ J(u) \in C^{2}(\mathbb{H}) $.  Докажите, что если при некотором $ \mu>0$ 
\begin{equation*}
J'(u_{*})=\Theta,\left\langle J''(u_{*}) h, h\right\rangle \geqslant \mu\|h\|_{\mathbb{H}}^{2} \quad \forall h \in \mathbb{H}
\end{equation*}
точка $  u_{*} $ является точкой минимума функционала $ J(u)$, а если существуют $ h_{1}, h_{2} \neq \Theta $ такие, что
\begin{equation*}
\left\langle J''\left(u_{*}\right) h_{1}, h_{1}\right\rangle>0,\left\langle J''\left(u_{*}\right) h_{2}, h_{2}\right\rangle<0
\end{equation*}
в точке $ u_{*} $ нет экстремума. \textbf{Указание:} используйте формулу Тейлора.\\
\\
$\boxed{\text{Решение.}}$ 1) Так как $J(u)\in C^2(\mathbb{H})$, то
\begin{equation*}
J(u_*+h)=J(u_*)+\langle J'(u_*),h\rangle+\frac{1}{2}\langle J''(u_*)h,h\rangle +o(\|h\|^2),\quad\forall h\in\mathbb{H}
\end{equation*}
В силу условия задачи получаем
\begin{equation*}
J(u_*+h)=J(u_*)+\frac{1}{2}\langle J''(u_*)h,h\rangle+o(\|h\|^2)\geqslant J(u_*)+\frac{\mu\|h\|^2}{2}+o(\|h\|^2)
\end{equation*}
При достаточном малом $\|h\|$ будет выполнено $\frac{\mu\|h\|^2}{2}+o(\|h\|^2)\geqslant 0$, поэтому $J(u_*+h)\geqslant J(u_*),\forall h\in \mathcal{O}(\varepsilon,u_*)$, где $\mathcal{O}(\varepsilon,u_*)$ - какая-то окрестность точки $u_*$ радиуса $\varepsilon$. А значит, $u_*$ - точка минимума.\\
2) Докажем что, $\langle J''(u_*)h,h\rangle\geqslant 0, \forall h\in\mathbb{H}$ - необходимое условие экстремума. Возьмем произвольный элемент $h\in\mathbb{H}$. В силу $J(u)\in C^2(\mathbb{H})$ имеем формулу Тейлора.
\begin{equation*}
J(u+e)-J(u)=\langle J'(u),e\rangle+\frac{1}{2}\langle J''(u)e,e\rangle +o(\|e\|^2),\quad\forall e\in\mathbb{H}
\end{equation*}
Положим $u=u_*, e=th,t\in\mathbb{R}^1$, и с учетом необходимого условия экстремума ($J'(u_*)=\Theta$) получаем
\begin{equation*}
\begin{aligned}
0\leqslant& J(u_*+th)-J(u_*)=\frac{1}{2}\langle J''(u_*)th,th\rangle+o(\|th\|^2)\\
0\leqslant& J(u_*+e)-J(u_*)=\frac{t^2}{2}\langle J''(u_*)h,h\rangle+o(t^2)
\end{aligned}
\end{equation*}
Деля $t^2$ при $t\ne 0$ и устремляя $t\rightarrow 0$ получаем
\begin{equation*}
\langle J''(u_*)h,h\rangle\geqslant 0, \forall h\in\mathbb{H}
\end{equation*}
Утверждение доказано. Поэтому если $\langle J''(u_*)h_2,h_2\rangle<0$, то необходимое условие нарушено, значит в точке $u_*$ нет экстремума.\\
$\boxed{\text{Ответ.}}$ Доказательства см. выше\\
\\
\textbf{3.2.} С помошью утверждения из задачи 3.1 найдите точки минимума функционала $ J: \ell^{2} \rightarrow \mathbb{R},$
\begin{equation*}
J(x)=\sum_{n=1}^{\infty}\left(x_{n}-\frac{1}{3^{n}}\right)^{2}+\left(\sum_{n=1}^{\infty} \frac{x_{n}}{2^{n}}\right)^{2}
\end{equation*}
$\boxed{\text{Решение.}}$ Запишем функционал в виде
\begin{equation*}
J(u)=\|u-f\|^2+\langle u,c\rangle^2,\quad f=(\frac{1}{3^1},\ldots,\frac{1}{3^n},\ldots),c=(\frac{1}{2^1},\ldots,\frac{1}{2^n},\ldots),u=(x_1,\ldots,x_n,\ldots)\in\ell^2
\end{equation*}
Теперь вычисляем градиент и гессиан функционала:
\begin{equation*}
\begin{aligned}
J'(u)=&2(u-f)+2\langle u,c\rangle c\\
J''(u)=&2\mathcal{E}+2\langle c,\cdot\rangle c
\end{aligned}
\end{equation*}
Заметим, что $\langle J''(u)h,h\rangle = \langle 2h,2\langle c,h\rangle c,h\rangle=2\|h\|^2+2\langle c,h\rangle^2\geqslant 0$, при этом $0$ достигается при $h=\Theta$. Поэтому выполнен критерий выпуклости 2-го порядка и значит функционал выпуклый, при этом также выполнено необходимое условие экстремума для гессиана во всех точках пространства. Далее, Из $J'(u_*)=\Theta$ получаем
\begin{equation*}
J'(u_*)=2(u_*-f)+2\langle u_*,c \rangle c=\Theta\quad\Rightarrow\quad u_*+\langle u_*,c\rangle c=f
\end{equation*}
Видно, что $u_*$ является линейной комбинацией элемента $f,c$, которые линейно независимы. В линейной независимости можно убедиться из простого рассуждения: пусть они линейно зависимы, тогда можно записать в виде $f=kc,k\in\mathbb{R}^1$. Это значит, что нужно выполнить все равенства $1/3^1=k\cdot 1/2^1,\ldots,1/3^n=k\cdot 1/2^n,\ldots$, что невозможно.\\
Поэтому положим $u_*=\lambda f+\mu c.$. Подставляя и получаем
\begin{equation*}
\begin{aligned}
\lambda f&+\mu c+\langle\lambda f+\mu c,c\rangle c=f\quad\Rightarrow \lambda f+\mu c+(\lambda\langle f,c\rangle+\mu\langle c,c\rangle )c=f\\
\Rightarrow&\{\textcolor{blue}{\langle f,c\rangle=\sum_{n=1}^{\infty}\frac{1}{6^n}=\frac{1}{5},\langle c,c\rangle=\sum_{n=1}^{\infty}\frac{1}{4^n}=\frac{1}{3}}\}\Rightarrow \lambda f+\mu c+\frac{1}{5}\lambda c+\frac{1}{3}\mu c =f\\
\Rightarrow&(\lambda-1)f+(\frac{4}{3}\mu+\frac{1}{5}\lambda)c=\Theta\\
&\left\{\begin{array}{cc}
\lambda-1=0\\
\frac{4}{3}\mu+\frac{1}{5}\lambda=0
\end{array}\right.\Rightarrow 
\left\{\begin{array}{cc}
\lambda=1\\
\mu=-\frac{3}{20}
\end{array}\right.\Rightarrow u_*=f-\frac{3}{20}c
\end{aligned}
\end{equation*}
Поскольку функционал выпуклый, то найденная точк $u_*$а является точкой минимума по критерию оптимальности.\\
$\boxed{\text{Ответ.}}$ $u_*=f-\frac{3}{20}c=(\frac{1}{3^1}-\frac{3}{20}\cdot\frac{1}{2^1},\ldots,\frac{1}{3^n}-\frac{3}{20}\cdot\frac{1}{2^n},\ldots)$\\
\textbf{4.} Исследуйте на выпуклость и сильную выпуклость функционал $ J(u)=\int\limits_{0}^{1}\left(\int\limits_{0}^{t^{3}} u(s) d s-t^{3}\right)^{2} d t: L^{2}(0 ; 1) \rightarrow \mathbb{R}.$
$\boxed{\text{Решение.}}$ Этот функционал можно записать в виде
\begin{equation*}
J(u)=\|\mathcal{A}u-f\|^2,\quad u=u(t)\in L^2(0;1),\mathcal{A}u(t)=\int_{0}^{t^3}u(s)ds,f(t)=t^3.
\end{equation*}
Покажем, что оператор линейный и ограниченный.
\begin{equation*}
\mathcal{A}(\alpha u+\beta v)=\textcolor{blue}{\{ \text{линейность интеграла}\}}=\alpha \int_{0}^{t^3}u(s)ds+\beta\int_{0}^{t^3}v(s)ds=\alpha\mathcal{A}u+\beta\mathcal{A}v
\end{equation*}
\begin{equation*}
\begin{aligned}
\|\mathcal{A}u\|^2=&\int_{0}^{1}\left(\int_{0}^{t^3}u(s)ds \right)^2dt\leqslant\int_{0}^{1}\left(\int_{0}^{t^3}1\cdot |u(s)|ds \right)^2dt\leqslant\textcolor{blue}{\{ \text{нер-во Гельдера}\}}\leqslant\int_{0}^{1}\left(\int_{0}^{t^3}1^2ds\int_{0}^{t^3}u^2(s)ds \right)dt\leqslant\\
\leqslant&\int_{0}^{1}t^3\int_{0}^{t^3}u^2(s)dsdt\leqslant\textcolor{blue}{\{ t^3\leqslant 1,u^2(s)\geqslant 0\}}\leqslant\int_{0}^{1}\int_{0}^{1}u^2(s)dsdt=\int_{0}^{1}\|u\|^2dt=\|u\|^2
\end{aligned}
\end{equation*}
Итак, функционал вида $\|\mathcal{A}u-f\|^2$ - выпуклый. Для сильной выпуклости необходимо и достаточно, чтобы существовал линейный ограниченный обратный оператор $\mathcal{A}^{-1}$. Для этого надо выполнять условие $2\|\mathcal{A}u\|^2\geqslant\mu\|u\|^2,\mu>0\in\mathbb{R}^1,\forall u\in L^2(0;1)$. Однако это неверно.\\
Рассмотрим последовательность функций $\{u_n(t) \}$
\begin{equation*}
u_n(t)=\left\{\begin{array}{cc}
1,\quad t\in\big[\sqrt[3]{\frac{i}{2n}};\sqrt[3]{\frac{i+1}{2n}}\big], i - $\text{четный}$\\
-1,\quad t\in\big[\sqrt[3]{\frac{i}{2n}};\sqrt[3]{\frac{i+1}{2n}}\big], i - $\text{нечетный}$\\
\end{array}\right.
\end{equation*}
Заметим, что $\|u_n\|^2=1$. Далее вычисляем $\|\mathcal{A}u_n\|^2$:
\begin{equation*}
\begin{aligned}
\|Au_n\|^2=&\int_{0}^{1}(\mathcal{A}u_n(t))^2dt\leqslant\textcolor{blue}{\{0\leqslant\mathcal{A}u_n(t)\leqslant 1\}}\leqslant\int_{0}^{1}\mathcal{A}u_n(t)dt\leqslant\overbrace{\big(\sqrt[3]{\frac{2}{2n}}+\sqrt[3]{\frac{2}{2n}}-\sqrt[3]{\frac{1}{2n}}\big)\cdot \frac{1}{2n}\cdot\frac{1}{2}}^{\text{площадь первой трапеции}}\cdot n=\\
=&\frac{2\sqrt[3]{2}-1}{\sqrt[3]{2n}}\cdot\frac{1}{4}=\frac{2\sqrt[3]{2}-1}{4\sqrt[3]{2}}\cdot\frac{1}{\sqrt[3]{n}}=\textcolor{blue}{\{\|u_n\|^2=1 \}}=\frac{\text{const}}{\sqrt[3]{n}}\|u_n\|^2\stackrel{n\rightarrow\infty}{\rightarrow}0\not\geqslant\mu\|u_n\|^2,\quad\mu>0\in\mathbb{R}^1.
\end{aligned}
\end{equation*}
Из картинки видно, что тот интеграл ($\int\limits_{0}^{1}\mathcal{A}u_n(t)dt$), который мы оцениваем, представляет собой площадь фигуры, ограниченной графика $\mathcal{A}u_n(t)$ и оси $x$. Площадь этой фигуры, конечно меньше площади фигуры, помеченной косыми линиями. А площадь фигуры с косыми линиями меньше чем площадь первой трапеции, умноженной на $n$.\\
Поэтому не выполнено $2\|\mathcal{A}u\|^2\geqslant\mu\|u\|^2,\mu>0\in\mathbb{R}^1,\forall u\in L^2(0;1)$ и функционал не сильно выпуклый.\\
 $\boxed{\text{Ответ.}}$ Функционал выпуклый, но не сильно выпуклый.\\
 \begin{figure}[htbp] 
 	\centering 
 	\begin{minipage}[t]{0.48\textwidth} 
 		\centering 
 		\includegraphics[width=7cm]{2p} 
 		\caption{График к задаче $1.3$}
 		\label{fig:1}
 	\end{minipage} 
 	\begin{minipage}[t]{0.48\textwidth} 
 			\centering
 		\includegraphics[width=9cm]{3p}
 		\caption{График к задаче $4$}
 		\label{fig:2}
 	\end{minipage} 
 \end{figure}
\end{document}

